\documentclass{article}

\usepackage[dvipdfm]{graphicx}
\usepackage[dvipdfm]{color}
\usepackage[dvipdfm]{hyperref}

\usepackage{a4}
\setlength{\parindent}{0cm}

\title{PM - A REDUCE Pattern Matcher}
\author{Kevin McIsaac \\
	The University of Western Australia \\
	 and \\
	 The RAND Corporation \\
	 kevin@wri.com}
\date{}


\begin{document}
\maketitle

PM is a general pattern matcher similar in style to those found in systems
such as SMP and Mathematica, and is based on the pattern matcher described
in Kevin McIsaac, \char`\"{}Pattern Matching Algebraic Identities\char`\"{},
SIGSAM Bulletin, 19 (1985), 4-13. \\
\ \\
The following is a description of its structure. \\
\ \\
A template is any expression composed of literal elements (e.g. \char`\"{}5\char`\"{},
\char`\"{}a\char`\"{} or \char`\"{}a+1\char`\"{}) and specially denoted pattern 
variables (e.g. ?a or ??b). Atoms beginning with `?' are called generic variables 
and match any expression. \\
\ \\
Atoms beginning with `??' are called multi-generic variables and match any
expression or any sequence of expressions including the null or empty
sequence. A sequence is an expression of the form `{[}a1, a2,...{]}'. When
placed in a function argument list the brackets are removed, i.e. f({[}a,1{]})
$->$ f(a,1) and f(a,{[}1,2{]},b) $->$ f(a,1,2,b). \\
\ \\
A template is said to match an expression if the template is literally
equal to the expression or if by replacing any of the generic or
multi-generic symbols occurring in the template, the template can be made
to be literally equal to the expression. These replacements are called the
bindings for the generic variables. A replacement is an expression of the
form `exp1 $->$ exp2', which means exp1 is replaced by exp2, or `exp1 $-->$
exp2', which is the same except exp2 is not simplified until after the
substitution for exp1 is made. If the expression has any of the
properties; associativity, commutativity, or an identity element, they are
used to determine if the expressions match. If an attempt to match the
template to the expression fails the matcher backtracks, unbinding generic
variables, until it reached a place were it can make a different choice.
It then proceeds along the new branch. \\
\ \\
The current matcher proceeds from left to right in a depth first search of
the template expression tree. Rearrangements of the expression are
generated when the match fails and the matcher backtracks. \\
\ \\
The matcher also supports semantic matching. Briefly, if a subtemplate
does not match the corresponding subexpression because they have different
structures then the two are equated and the matcher continues matching the
rest of the expression until all the generic variables in the subexpression
are bound. The equality is then checked. This is controlled by the switch
`semantic'. By default it is on. \\

\pagebreak

{\tt M($exp,temp$)} \\
\ \\
\begin{tabular}{lp{11cm}}
\hspace*{0.2cm} & The template, temp, is matched against the expression, exp. If the
template is literally equal to the expression `T' is returned. If the
template is literally equal to the expression after replacing the
generic variables by their bindings then the set of bindings is returned
as a set of replacements. Otherwise 0 (nil) is returned. \\
\end{tabular} \\
\ \\
\ \\
{\bf Examples:} \\
\ \\


 A \char`\"{}literal\char`\"{} template

 m(f(a),f(a));

 T



 Not literally equal

 m(f(a),f(b));

 0



 Nested operators

 m(f(a,h(b)),f(a,h(b)));

 T



 a \char`\"{}generic\char`\"{} template

 m(f(a,b),f(a,?a));

 \{?A$->$B\}



 m(f(a,b),f(?a,?b));

 \{?B$->$B,?A$->$A\}



 The Multi-Generic symbol, ??a, takes \char`\"{}rest\char`\"{} of arguments

 m(f(a,b),f(??a));

 \{??A$->${[}A,B{]}\}



 but the Generic symbol, ?a, does not

 m(f(a,b),f(?a));

 0



 Flag h as associative

 flag('(h),'assoc);

 Associativity is used to \char`\"{}group\char`\"{} terms together

 m(h(a,b,d,e),h(?a,d,?b));

 \{?B$->$E,?A'$->$H(A,B)\}



 \char`\"{}plus\char`\"{} is a symmetric function

 m(a+b+c,c+?a+?b);

 \{?B$->$A,?A$->$B\}



 it is also associative

 m(a+b+c,b+?a);

 \{?A$->$C + A\}



 Note the affect of using multi-generic symbol is different

 m(a+b+c,b+??c);

 \{??C$->${[}C,A{]}\}



temp \_= logical-exp \\
\ \\
A template may be qualified by the use of the conditional operator `\_=',
such!-that. When a such!-that condition is encountered in a template it
is held until all generic variables appearing in logical-exp are bound.

On the binding of the last generic variable logical-exp is simplified
and if the result is not `T' the condition fails and the pattern matcher
backtracks. When the template has been fully parsed any remaining held
such-that conditions are evaluated and compared to `T'. \\
\ \\
{\bf Examples:} \\
\ \\
 m(f(a,b),f(?a,?b\_=(?a=?b)));

 0


 m(f(a,a),f(?a,?b\_=(?a=?b)));

 \{?B$->$A,?A$->$A\}



 Note that f(?a,?b\_=(?a=?b)) is the same as f(?a,?a)



S(exp,\{temp1$->$sub1,temp2$->$sub2,...\},rept, depth) \\
\ \\
Substitute the set of replacements into exp, resubstituting a maximum of
'rept' times and to a maximum depth 'depth'. 'Rept' and 'depth' have the
default values of 1 and infinity respectively. Essentially S is a
breadth first search and replace.

Each template is matched against exp until a successful match occurs.

Any replacements for generic variables are applied to the rhs of that
replacement and exp is replaced by the rhs. The substitution process is
restarted on the new expression starting with the first replacement. If
none of the templates match exp then the first replacement is tried
against each sub-expression of exp. If a matching template is found
then the sub-expression is replaced and process continues with the next
sub-expression.

When all sub-expressions have been examined, if a match was found, the
expression is evaluated and the process is restarted on the
sub-expressions of the resulting expression, starting with the first
replacement. When all sub-expressions have been examined and no match
found the sub-expressions are reexamined using the next replacement.
Finally when this has been done for all replacements and no match found
then the process recures on each sub-expression.


The process is terminated after rept replacements or when the expression
no longer changes.



Si(exp,\{temp1$->$sub1,temp2$->$sub2,...\}, depth)


Substitute infinitely many times until expression stops changing.
Short hand notation for S(exp,\{temp1$->$sub1,temp2$->$sub2,...\},Inf,
depth)


Sd(exp,\{temp1$->$sub1,temp2$->$sub2,...\},rept, depth)


Depth first version of Substitute.\\
\ \\
{\bf Examples:} \\
\ \\
 s(f(a,b),f(a,?b)$->$?b\^{}2);

 2

 B



 s(a+b,a+b$->$a{*}b);

 B{*}A



 \char`\"{}associativity\char`\"{} is used to group a+b+c in to (a+b)
+ c

 s(a+b+c,a+b$->$a{*}b);

 B{*}A + C


The next three examples use a rule set that defines the factorial function.

Substitute once

 s(nfac(3),\{nfac(0)$->$1,nfac(?x)$->$?x{*}nfac(?x-1)\});

 3{*}NFAC(2)


Substitute twice

 s(nfac(3),\{nfac(0)$->$1,nfac(?x)$->$?x{*}nfac(?x-1)\},2);

 6{*}NFAC(1)


Substitute until expression stops changing

 si(nfac(3),\{nfac(0)$->$1,nfac(?x)$->$?x{*}nfac(?x-1)\});

 6


Only substitute at the top level

 s(a+b+f(a+b),a+b$->$a{*}b,inf,0);

 F(B + A) + B{*}A


temp :- exp \\
\ \\
If during simplification of an expression, temp matches some
sub-expression then that sub-expression is replaced by exp. If there is
a choice of templates to apply the least general is used.

If a old rule exists with the same template then the old rule is
replaced by the new rule. If exp is `nil' the rule is retracted.



temp ::- exp


Same as temp :- exp, but the lhs is not simplified until the replacement
is made \\
\ \\
{\bf Examples:} \\
\ \\
Define the factorial function of a natural number as a recursive function
and a termination condition. For all other values write it as a Gamma
Function. Note that the order of definition is not important as the rules
are reordered so that the most specific rule is tried first.

Note the use of `::-' instead of `:-' to stop simplification of
the LHS. Hold stops its arguments from being simplified. \\
\ \\
 fac(?x\_=Natp(?x)) ::- ?x{*}fac(?x-1);

 HOLD(FAC(?X-1){*}?X)



 fac(0) :- 1;

 1



 fac(?x) :- Gamma(?x+1);

 GAMMA(?X + 1)



 fac(3);

 6



 fac(3/2);

 GAMMA(5/2)



Arep(\{rep1,rep2,..\}) \\
\ \\
In future simplifications automatically apply replacements rep1,
rep2...~ until the rules are retracted. In effect it replaces the
operator `$->$' by `:-' in the set of replacements \{rep1, rep2,...\}.



Drep(\{rep1,rep2,..\})



 Delete the rules rep1, rep2,... \\
\ \\
As we said earlier, the matcher has been constructed along the lines of the
pattern matcher described in McIsaac with the addition of such-that
conditions and `semantic matching' as described in Grief. To make a
template efficient some consideration should be given to the structure of
the template and the position of such-that statements. In general the
template should be constructed to that failure to match is recognize as
early as possible. The multi-generic symbol should be used when ever
appropriate, particularly with symmetric functions. For further details
see McIsaac. \\
\ \\
{\bf Examples:} \\
\ \\
 f(?a,?a,?b) is better that f(?a,?b,?c\_=(?a=?b))

 ?a+??b is better than ?a+?b+?c...

The template, f(?a+?b,?a,?b), matched against f(3,2,1) is
matched as f(?e\_=(?e=?a+?b),?a,?b) when semantic matching is allowed. \\

{\bf Switches} \\
\ \\
{\tt TRPM} \\
Produces a trace of the rules applied during a substitution. This is
useful to see how the pattern matcher works, or to understand an
unexpected result. \\
\ \\
In general usage the following switches need not be considered. \\
\ \\
{\tt SEMANTIC} \\
Allow semantic matches, e.g. f(?a+?b,?a,?b) will match f(3,2,1) even
though the matcher works from left to right. \\
\ \\
{\tt SYM!-ASSOC} \\
Limits the search space of symmetric associative functions when the
template contains multi-generic symbols so that generic symbols will not
the function. For example: m(a+b+c,?a+??b) will return \{?a $->$ a, ??b$->$
{[}b,c{]}\} or \{?a $->$ b, ??b$->$ {[}a,c{]}\} or \{?a $->$ c, ??b$->$ {[}a,b{]}\}
but no \{?a $->$ a+b, ??b$->$ c\} etc. No sane template should require these 
types of matches. However they can be made available by turning the switch off.

\end{document}
