\documentclass[12pt]{article}
%Sets size of page and margins
\textheight 21cm
\textwidth 15cm

\title{Manual for CONLAW versions 1,2,3,4}
\author{Thomas Wolf \\                        
        Department of Mathematics \\
        Brock University \\
        St.Catharines \\
        Ontario, Canada L2S 3A1 \\
        twolf@brocku.ca}

\begin{document}
\maketitle
\section{Purpose}
The procedures {\tt CONLAW1, CONLAW2, CONLAW3, CONLAW4} 
try to find conservation laws
for a given single/system of differential equation(s) (ODEs or PDEs)
\begin{equation}
     u^{\alpha}_J = w^{\alpha}(x,u^{\beta},...,u^{\beta}_K,...)  \label{a1}
\end{equation}
{\tt CONLAW1} tries to find the conserved current $P^i$ by solving
\begin{equation}
   \mbox{Div}\;\,P = 0  \;\;\; \mbox{modulo}\;\; (\ref{a1})
\label{a2}
\end{equation}
directly. {\tt CONLAW3} tries to find $P^i$ and the characteristic
functions (integrating factors) $Q^{\nu}$ by solving
\begin{equation}
\mbox{Div}\,P = \sum_{\nu} Q^{\nu}\cdot (u^{\nu}_J - w^{\nu}) \label{a3}
\end{equation}
identically in all $u$-derivatives.
Applying the Euler operator (variational derivative) for each $u^{\nu}$
on (\ref{a3}) gives a zero left hand side and therefore
conditions involving only $Q^{\nu}$. {\tt CONLAW4} tries to solve these
conditions identically in all $u$-derivatives and to compute $P^i$
afterwards. {\tt CONLAW2}
does substitutions based on (\ref{a1}) before solving these
conditions on $Q^{\nu}$
and therefore computes adjoined symmetries. These are
completed, if possible, to conservation laws by computing $P^i$ from
the $Q^{\nu}$.

All four procedures have the same syntax. They have two parameters,
both are lists. The first parameter specifies the equations (\ref{a1}), 
the second specifies the computation to be done. One can either specify an 
ansatz for $P^i, Q^{\nu}$ or investigate a general situation, only specifying 
the order of the characteristic functions or the conserved current. For a 
more detailed description see the file {\tt conca.tex}.

The file {\tt CONLAW0.RED} contains subroutines used in all four versions.

\section{The Syntax}
The procedures {\tt CONLAWi} $\;\;$ i=1,2,3,4 are called through \\
{\tt CONLAWi({\it problem,runmode}); } \\
where i=1,2,3,4. Both parameters {\it problem, runmode} are lists.
The first specifies the DEs to be investigated: \\
{\it problem} ... \{{\it equations, ulist, xlist}\}
\begin{tabbing}
\hspace{0.5cm}
 {\it equations}\= ... \= list of equations,
              each has the form {\tt df(ui,..)=...} where \\
       \>  \> the LHS (left hand side) {\tt df(ui,..)} is selected such that \\
       \>  \>  - The RHS (right h.s.) of an equations must not include     \\
       \>  \>$\;\;$ the derivative on the LHS nor a derivative of it.  \\
       \>  \>  - The LHS of any equation should not occur in any other\\
       \>  \>$\;\;$ equation nor any derivative of the LHS.  \\
       \>  \> If CONLAW3 or CONLAW4 are run where no substitutions\\
       \>  \> are made then the LHS of equations can be {\tt df(ui,..)**n=...}\\
       \>  \> where {\tt n} is a number. No difference is made between\\
       \>  \> equations and constraints. \\
\hspace{0.5cm}
 {\it ulist} \>  ... \> list of function names, which can be chosen freely \\
             \>      \> the number of functions and equations need not
be equal \\
\hspace{0.5cm}
 {\it xlist}  \>  ... \> list of variable names, which can be chosen freely 
\end{tabbing}
The second parameter specifies the calculation to be done.  \\
{\it runmode} ... \{{\it minord, maxord, expl, flist, inequ}\}
\begin{tabbing}
\hspace{0.5cm}
     {\it minord} \= ... \= the minimum of the highest order of
                          derivatives of u \\
             \> \>        - in {\tt p\_t} for CONLAW1 where {\tt t} is the first variable in\\
             \> \>        \ \  {\tt xlist} and \\
             \> \>        - in {\tt q\_j} for CONLAW2,CONLAW3,CONLAW4  \\
\hspace{0.5cm}
     {\it maxord} \> ... \> the maximum of the highest order of 
                          derivatives of u \\
             \> \>        - in {\tt p\_i} for CONLAW1 where {\tt t} is the first variable in\\
             \> \>        \ \  {\tt xlist} and  \\
             \> \>        - in {\tt q\_j} for CONLAW2,CONLAW3,CONLAW4 \\
\hspace{0.5cm}
     {\it expl}       \> ... \> (t/nil) whether or not the charac.\ 
                          functions {\tt q\_i} or conserved \\
             \> \>        current may depend explicitly on
                          the variables of xlist \\
\hspace{0.5cm}
     {\it flist}      \> ... \> a list of unknown functions in any ansatz for
                          {\tt p\_i,q\_j}, \\
        \>  \>            also all parameters and parametric functions in
                          the equation \\
        \>  \>            that are to be calculated such that
                          conservation laws exist,\\
        \>  \>            if there are no such unknown functions then
                          {\it flist} is the \\
        \>  \>            empty list: \verb+{}+ \\
\hspace{0.5cm}
     {\it inequ}      \> ... \> a list of expressions non of which
                          may be identically \\ 
        \>  \>            zero for the conservation
                          law to be found, if there is no such \\
        \>  \>            expression
                          then {\it inequ} is an empty list: \verb+{}+
\end{tabbing}
The procedures {\tt CONLAWi} return a list of conservation laws
\verb+ {+$C_1,C_2,\ldots$\verb+}+, if no
non-trivial conservation law is found they return the empty list \verb+{}+.
Each $C_i$ representing a conservation law
has the form  \verb+ {{+$P^1,P^2,\ldots$\verb+},{+$Q^1,Q^2,\ldots$\verb+}}+.

An ansatz for a conservation law can be formulated by specifying one
or more of the components $P^i$ for {\tt CONLAW1}, one or more of the
functions $Q^{\mu}$ for {\tt CONLAW2, CONLAW4} or one or more of 
$P^i, Q^{\mu}$ for {\tt CONLAW3}. The $P^i$ are input as {\tt p\_i} where
{\tt i} in {\tt p\_i} stands for a variable name, and the $Q^{\mu}$ are
input as {\tt q\_i} where {\tt i} stands for an index - the number of the
equation in the input list {\it equations} with which {\tt q\_i} is
multiplied.

There is a restriction in the structure of all the expressions for
{\tt p\_i, q\_j} that are specified: they must be homogeneous linear
in the
unknown functions or constants which appear in these expressions. 
The reason for this restriction is not
for CRACK to be able to solve the resulting overdetermined system but for
{\tt CONLAWi} to be able afterwards to extract the individual conservation
laws from the general solution of the determining conditions.

All such unknown functions and constants must be
listed in {\it flist} (see above). The dependencies of such functions
must be defined before calling {\tt CONLAWi}. This is done with the
command {\tt DEPEND}, e.g. {\tt DEPEND f,t,x,u\$ } to specify $f$ as a
function of $t,x,u$. If one wants to have $f$ as a function of derivatives
of $u(t,x)$, say $f$ depending on $u_{txx}$, then one can {\it not} write \\
\verb+    DEPEND f,df(u,t,x,2)$ + \\ 
but instead \\
\verb+    DEPEND f,u!`1!`2!`2$ + \\
if {\it xlist} has been specified as \verb+ {t,x}+, because {\tt t} is the
first variable and {\tt x} is the second variable in {\it xlist} and {\tt u}
is differentiated ones wrt.\ {\tt t} and two times wrt.\ {x} we therefore get
\verb+ u!`1!`2!`2+. The character {\tt !} is the exempt character to allow
special characters like ` to occur in an identifier name. 

It is possible to add extra conditions like PDEs for $P^i, Q^{\mu}$ as a list 
{\tt cl\_condi} of expressions that shall vanish. \hspace{6pt}
\noindent Remarks: 
\begin{itemize}
\item The input to {\tt CONLAW1, CONLAW2, CONLAW3, CONLAW4} is the 
same apart from:
\begin{itemize}
\item an ansatz for $Q^{\nu}$ is ignored in {\tt CONLAW1}
\item an ansatz for $P^i$ is ignored in {\tt CONLAW2, CONLAW4}
\item the meaning of mindensord, maxdensord is different in {\tt CONLAW1} 
on one hand and {\tt CONLAW2,CONLAW3,CONLAW4} on the other (see above).
\end{itemize}

\item
It matters how the differential equations are input, i.e. which
derivatives are eliminated. For example, the Korteweg - de Vries
equation if input in the form $u_{xxx}=-uu_x-u_t$ instead of
$u_t=-uu_x-u_{xxx}$
in {\tt CONLAW1} and choosing {\tt maxdensord}=1 then
$P^i$ will be of at most first order, Div $P$ of second order and $u_{xxx}$
will not be substituted and no non-trival conservation laws can be found.
This does not mean that one will not find low order conservation laws at
all with the substitution $u_{xxx}$ one only has to go to 
{\tt maxdensord}=2 with longer computations as a consequence compared to the 
input $u_t=-uu_x-u_{xxx}$ where {\tt maxdensord}=0 is enough to find 
non-trivial conservation laws.

\item The drawback of using $u_t=\ldots$ compared with $u_{xxx}=\ldots$
is that when seeking all conservation laws up to some order
then one has to investigate a higher order ansatz, because with each
substitution $u_t=-u_{xxx}+\ldots$ the order increases by 2.
For example, if all conservation laws of order up to two in $Q^{\nu}$ 
are to be determined then in order to include a $u_{tt}$-dependence the 
dependence of $Q^{\nu}$ on $u_x$ up to $u_{6x}$ has to be considered.

\item Although for any equivalence class of conserved currents $P^i$
differing only by a curl, there exist characteristic functions $Q^{\mu}$,
this need not be true for any particular $P^i$.
Therefore one cannot specify a known density $P^i$ for {\tt CONLAW3}
and hope to calculate the remaining $P^j$ and the corresponding $Q^{\mu}$
with {\tt CONLAW3}. What one
can do is to use {\tt CONLAW1} to calculate the remaining components 
$P^j$ and from this a trivial conserved
density $R$ and characteristic functions $Q^{\nu}$ are computed such that
\[ \mbox{Div}\,(P-R) = \sum_{\nu} Q^{\nu}\cdot (u^{\nu}_J - w^{\nu}). \]

\item The $Q^{\mu}$ are uniquely determined only modulo $\Delta=0$.
If one makes an ansatz for $Q^{\mu}$ then this freedom should be removed
by having
the $Q^{\mu}$ independent of the LHS's of the equations and independent of
derivatives of the LHS's of them. If the $Q^{\mu}$ were allowed to depend on
anything, then (\ref{a3}) could simply be solved for one $Q^{\nu}$ in terms of
arbitrary $P^j$ and other arbitrary $Q^{\rho}$, giving $Q^{\nu}$ that are
singular for solutions of the differential equations as the expression
of the differential equation would appear in the denominator of $Q^{\nu}$.

\item Any ansatz for $P^i,Q^{\nu}$ should
as well be independent of the LHS's of the equations (\ref{a1}) and 
independent of derivatives of the LHS's of (\ref{a1}).

\item If in equation (\ref{a3}) the right hand side is of order $m$ then
from the conserved current $P^i$ a trivial conserved current can be
subtracted such that the remaining conserved current is at most of order $m$.
If the right hand side is linear in the highest derivatives of order
$m$ then subtraction of a trivial conserved current can even achieve a
conserved current of order $m-1$. The relevance of this result is that
if the right hand side is known to be linear in the highest derivatives
then for $P^i$ an ansatz of order $m-1$ is only necessary. To take
advantage of this relation
if the right hand side is known to be linear in the highest
derivatives, a flag {\tt quasilin\_rhs} can be set to {\tt t} (see
below). 

\end{itemize}

\section{Flags, parameters}

\verb+    LISP (PRINT_:= NIL/0/1/ ...)$+ \\
\verb+print_=nil+ suppresses all CRACK output, for \verb+print_=n+ 
($n$ an integer) 
{\tt CRACK} prints only equations with at most $n$ factors in its terms.

%\verb+ LISP (LOGOPRINT_:=t/nil)$+ \\
%to print/not print a logo at the start of {\tt CONLAWi}

\verb+ CRACKHELP()$+ \\
to show other flags controling the solution of the overdetermined PDE-system,

\verb+ OFF BATCH_MODE$+ \\
to solve the system of conditions with {\tt CRACK} interactively.

\verb+ LISP(QUASILIN_RHS:=T)$+ \\
reduces in the ansatz for $P^i$ the order to $m-1$ if the order of the
right hand side is $m$. This can be used to speed up computations if
the right hand side is known to be linear in the highest derivatives
(see the note above).

\section{Requirements}

{\tt REDUCE 3.6} and 
the files {\tt CRACK.RED, CONLAW0.RED}, one of the files
{\tt CONLAW1.RED, CONLAW2.RED, CONLAW3.RED, CONLAW4.RED} depending which
program should be used and all files {\tt CR*.RED} which 
are read in from {\tt CRACK.RED}.

One either has to read in files with \\
\verb+ IN "crack.red","conlaw0.red","conlaw1.red"$+ \\
\noindent (and appropriate paths) or compile them before with

\begin{verbatim}
    FASLOUT "crack"$
    IN "crack"$
    FASLEND$
    FASLOUT "conlaw0"$
    IN "conlaw0.red"$
    FASLEND$
    FASLOUT "conlaw1"$
    IN "conlaw1.red"$
    FASLEND$
    BYE$
\end{verbatim}
and load them afterwards with \verb+ LOAD crack,conlaw0,conlaw1$+ \\
\verb+conlaw2, conlaw3, conlaw4+ are treated like {\tt conlaw1}.

\section{Examples}
Below a {\tt CRACK}-procedure {\tt nodepnd} is used to clean up 
after each run and delete all dependencies of each 
function in the list of functions in the argument of {\tt nodepnd}.
More details concerning these examples are given when running
the file {\tt conlaw.tst}.

\verb+ lisp(print_:=nil);  +   to suppress output from CRACK 
\begin{itemize}
\item a single PDE:
\begin{verbatim}
   depend u,x,t$
   conlaw1({{df(u,t)=-u*df(u,x)-df(u,x,3)}, {u}, {t,x}},
           {0, 1, t, {}, {}})$
   nodepnd {u}$
\end{verbatim}

\item a system of equations:
\begin{verbatim}
   depend u,x,t$
   depend v,x,t$
   conlaw1({{df(u,t)=df(u,x,3)+6*u*df(u,x)+2*v*df(v,x),
             df(v,t)=2*df(u,x)*v+2*u*df(v,x)           }, 
            {u,v}, {t,x}},
           {0, 1, t, {}, {}})$
   nodepnd {u,v}$
\end{verbatim}

\item a system of equations with ansatz:
\begin{verbatim}
   depend u,x,t$
   depend v,x,t$
   depend r,t,x,u,v,u!`2,v!`2$
   q_1:=r*df(u,x,2)$
   conlaw2({{df(u,t)=df(v,x),
             df(v,t)=df(u,x) }, {u,v}, {t,x}},
           {2, 2, t, {r}, {r}})$
   nodepnd {u,v,r}$
\end{verbatim}

\item for the 
determination of parameters, such that conservation laws exist:
\begin{verbatim}
   depend u,x,t;
   conlaw1({{df(u,t)=-df(u,x,5)-a*u**2*df(u,x)
                     -b*df(u,x)*df(u,x,2)-c*u*df(u,x,3)}, 
            {u}, {t,x}},
           {0, 1, t, {a,b,c}, {}});
   nodepnd {u};
\end{verbatim}

\item for first integrals of an ODE-system including the determination of
parameter values s,b,r such that conservation laws exist:
\begin{verbatim}
   depend {x,y,z},t;
   depend a1,x,t;
   depend a2,y,t;
   depend a3,z,t;

   p_t:=a1+a2+a3;
   conlaw2({{df(x,t) = - s*x + s*y,
             df(y,t) = x*z + r*x - y,
             df(z,t) = x*y - b*z},
            {x,y,z},{t}
           },
           {0,0,t,{a1,a2,a3,s,r,b},{}});
   nodepnd {x,y,z,a1,a2,a3};
\end{verbatim}

\end{itemize}

\end{document}
