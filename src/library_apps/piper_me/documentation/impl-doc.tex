\begin{document}

\newtheorem{atheorem}{Theorem}[section]
\newtheorem{adefinition}{Definition}[section]
\newtheorem{analgo}{Algorithm}[section]
\newtheorem{anexample}{Example}[section]

\newtheorem{acorollary}[atheorem]{Korollar}
\newtheorem{asubtheo}[atheorem]{Lemma}
\newtheorem{atinytheo}[atheorem]{Proposition}
\newcommand{\Path}{{\rm Path}}
\newcommand{\LO}{{\rm LO}}
\newcommand{\Abl}{\mbox{\rm Abl}}
\newcommand{\LAST}{{\rm fin}}
\newcommand{\DIST}{\mbox{\rm DIST}}
\newcommand{\Root}{{\rm root}}
\newcommand{\Front}{\mbox{\rm front}}
\newcommand{\KOST}{{\cal C}}
\newcommand{\Choose}{{\rm Choose}}
\newcommand{\Eword}{\Box}
\newcommand{\Pbegin}{{\rm\bf begin }}
\newcommand{\Pprocedure}{{\rm\bf procedure }}
\newcommand{\Pfunction}{{\rm\bf function }}
\newcommand{\Pinteger}{{\rm\bf integer}}
\newcommand{\Pend}{{\rm\bf end }}
\newcommand{\Pfor}{{\rm\bf for }}
\newcommand{\Pforall}{{\rm\bf for all }}
\newcommand{\Pendfor}{{\rm\bf endfor }}
\newcommand{\Pto}{{\rm\bf to }}
\newcommand{\Pdownto}{{\rm\bf downto }}
\newcommand{\Pdo}{{\rm\bf do }}
\newcommand{\Pif}{{\rm\bf if }}
\newcommand{\Pthen}{{\rm\bf then }}
\newcommand{\Pendif}{{\rm\bf endif }}
\newcommand{\Pelse}{{\rm\bf else }}
\newcommand{\Pendelse}{{\rm\bf endelse }}
\newcommand{\Preturn}{{\rm\bf return }}
\newcommand{\Real}{{\rm{I\hspace*{-0.4ex}R}}}
\newcommand{\Nat}{{\rm{I\hspace*{-0.4ex}N}}}
\newcommand{\Bool}{{\rm{I\hspace*{-0.4ex}B}}}
\newcommand{\Uint}{{\rm{I\hspace*{-0.5ex}I}}}
\newcommand{\Zet}{{\rm\sf Z\hspace*{-1.0ex}Z}}
\newcommand{\QuestEq}{\stackrel{?}{=}}
\newcommand{\DefEq}{:=}
\newcommand{\DisUnion}{\stackrel{.}{\cup}}
\newcommand{\IsPref}{\le_{\mbox{\tiny pr"af}}}


\title{
{\LARGE Piper-ME and XPiper}\\
}

\maketitle

\chapter{Piper-ME and X-Piper}

Piper-ME is a port of Yacas to the Java Mobile Edition platform, CLDC version 1.1 
(that is, including double precision floating point numbers). The port is, however,
not finished yet. Most classes from the standard libraries which are nor part of
Java ME have been replaced by simplified reimplementation (in particular hashmaps
and extensdible arrays) or taken from the classpath project (big numbers). Only
console and file system based classes are not ported yet.\\
\\
X-Piper is planned to become a reimplementation of Piper/Yacas, which might
not be compatible. It is targeted to resolve the following issues:
\begin{enumerate}
\item Scoping of local symbols.
\item Multiple precision arithmetic.
\end{enumerate}

\chapter{Interpreter}

\section{Programming Concepts}

The development of the X-Piper interpreter is a work in progress. It is implemented
in such way that experimentation with different language features should be easy.
As a result, modules that deal with the syntactical structure of the language will
be developed at later. In order to test language features without being able to
parse source code, several interpreter modules are implemented as libraries which
are accessible from Java-code:
\begin{enumerate}
\item Construction of abstract syntax trees (ASTs).
\item Rulebase and scoping.
\item Evaluation of functions.
\end{enumerate}

\section{Tokenizer and Parser}

\chapter{Expressions and Abstract Syntax Trees}

Internally, an expression can be one of the folliwing things:

\begin{enumerate}
\item A variable, including operators and function symbols
\item A constant number or string.
\item An assignment of a variable or an extension of a rule (will be explained later).
\item A list of expressions.
\end{enumerate}

Syntactically, there might be some variations, especially of blocks, but that is 
now unimportant. 

\input{src4tex/Var.java}


\chapter{Evaluation}

\section{Frames and Rule Builder}

\input{src4tex/Frame.java}
\input{src4tex/RuleDefNode.java}

\chapter{Data Structures}

\section{Helper Data Structures}


\section{List Structure}

Lists are represented by a list structure taken from {\em eninom.lib}. The capabiliy
of dealing with lazy values is not used in Piper-ME.

We introduce CList:
 
\input{src4tex/CList.java}

\chapter{Large Numbers}
\input{src4tex/BigNumber.java}



\end{document}
