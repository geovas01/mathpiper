\begin{document}

\newtheorem{atheorem}{Theorem}[section]
\newtheorem{adefinition}{Definition}[section]
\newtheorem{analgo}{Algorithm}[section]
\newtheorem{anexample}{Example}[section]

\newtheorem{acorollary}[atheorem]{Korollar}
\newtheorem{asubtheo}[atheorem]{Lemma}
\newtheorem{atinytheo}[atheorem]{Proposition}
\newcommand{\Path}{{\rm Path}}
\newcommand{\LO}{{\rm LO}}
\newcommand{\Abl}{\mbox{\rm Abl}}
\newcommand{\LAST}{{\rm fin}}
\newcommand{\DIST}{\mbox{\rm DIST}}
\newcommand{\Root}{{\rm root}}
\newcommand{\Front}{\mbox{\rm front}}
\newcommand{\KOST}{{\cal C}}
\newcommand{\Choose}{{\rm Choose}}
\newcommand{\Eword}{\Box}
\newcommand{\Pbegin}{{\rm\bf begin }}
\newcommand{\Pprocedure}{{\rm\bf procedure }}
\newcommand{\Pfunction}{{\rm\bf function }}
\newcommand{\Pinteger}{{\rm\bf integer}}
\newcommand{\Pend}{{\rm\bf end }}
\newcommand{\Pfor}{{\rm\bf for }}
\newcommand{\Pforall}{{\rm\bf for all }}
\newcommand{\Pendfor}{{\rm\bf endfor }}
\newcommand{\Pto}{{\rm\bf to }}
\newcommand{\Pdownto}{{\rm\bf downto }}
\newcommand{\Pdo}{{\rm\bf do }}
\newcommand{\Pif}{{\rm\bf if }}
\newcommand{\Pthen}{{\rm\bf then }}
\newcommand{\Pendif}{{\rm\bf endif }}
\newcommand{\Pelse}{{\rm\bf else }}
\newcommand{\Pendelse}{{\rm\bf endelse }}
\newcommand{\Preturn}{{\rm\bf return }}
\newcommand{\Real}{{\rm{I\hspace*{-0.4ex}R}}}
\newcommand{\Nat}{{\rm{I\hspace*{-0.4ex}N}}}
\newcommand{\Bool}{{\rm{I\hspace*{-0.4ex}B}}}
\newcommand{\Uint}{{\rm{I\hspace*{-0.5ex}I}}}
\newcommand{\Zet}{{\rm\sf Z\hspace*{-1.0ex}Z}}
\newcommand{\QuestEq}{\stackrel{?}{=}}
\newcommand{\DefEq}{:=}
\newcommand{\DisUnion}{\stackrel{.}{\cup}}
\newcommand{\IsPref}{\le_{\mbox{\tiny pr"af}}}


\title{
{\LARGE Piper B}\\
}

\maketitle


\chapter{Piper-B for J2ME}

\section{Introduction}

Piper-B for J2ME is a port of Yacas/Piper to the J2ME environment and this way
make Yacas available on a multitude of mobile devices. The port has the
following goals:
\begin{enumerate}
\item To do math on a handheld.
\item To have an interesting general purpose programming language available.
\item To gain a better understanding of the current implementation of Yacas.
\item To improve Yacas.
\item To gain enough knowledge about Yacas semantics in order to provide
a cleanroom reimplementation, publishable under LGPL.
\end{enumerate}

\subsection{Platform}

The port to J2ME is still work in progress. The bulk of the code is written
for the intersection of J2ME and Java 1.5 with one notable exception: We
use the Java 1.5 memory model. That means that for Java 1.4 and older
platforms, concurrent excecution of the Yacas interpreter might not work
on multicore processors.\\
A regretable part is the lack of generics in J2ME. The readability of
the code suffers a lot. It would be nice to have set up a toolchain that
eliminates generics either from sources or the generated class files,
so that we can compile generic code for J2ME.

\subsection{What has been done?}

\begin{enumerate}
\item Immutable Interface for Yacas/Piper numbers
\item Integrating the next-references into LIspObjects. This saves memory.
In the long run, list structures should be explicit. An attempt for
efficient lists with Yacas semantics has been made in X-Piper's CoreList class.
\item The synchronized java.util.vector has been replaced by unsynchronized
structured.
\end{enumerate}


\chapter{Experimental Piper}

\section{X-Piper Core}

X-Piper is the experimental reimplementation of Piper-B/Yacas.

\subsection{The Core List Class}

\input{src4tex/CoreList.java}

\end{document}
