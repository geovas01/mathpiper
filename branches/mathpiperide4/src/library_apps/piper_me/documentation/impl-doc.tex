\begin{document}

\newtheorem{atheorem}{Theorem}[section]
\newtheorem{adefinition}{Definition}[section]
\newtheorem{analgo}{Algorithm}[section]
\newtheorem{anexample}{Example}[section]

\newtheorem{acorollary}[atheorem]{Korollar}
\newtheorem{asubtheo}[atheorem]{Lemma}
\newtheorem{atinytheo}[atheorem]{Proposition}
\newcommand{\Path}{{\rm Path}}
\newcommand{\LO}{{\rm LO}}
\newcommand{\Abl}{\mbox{\rm Abl}}
\newcommand{\LAST}{{\rm fin}}
\newcommand{\DIST}{\mbox{\rm DIST}}
\newcommand{\Root}{{\rm root}}
\newcommand{\Front}{\mbox{\rm front}}
\newcommand{\KOST}{{\cal C}}
\newcommand{\Choose}{{\rm Choose}}
\newcommand{\Eword}{\Box}
\newcommand{\Pbegin}{{\rm\bf begin }}
\newcommand{\Pprocedure}{{\rm\bf procedure }}
\newcommand{\Pfunction}{{\rm\bf function }}
\newcommand{\Pinteger}{{\rm\bf integer}}
\newcommand{\Pend}{{\rm\bf end }}
\newcommand{\Pfor}{{\rm\bf for }}
\newcommand{\Pforall}{{\rm\bf for all }}
\newcommand{\Pendfor}{{\rm\bf endfor }}
\newcommand{\Pto}{{\rm\bf to }}
\newcommand{\Pdownto}{{\rm\bf downto }}
\newcommand{\Pdo}{{\rm\bf do }}
\newcommand{\Pif}{{\rm\bf if }}
\newcommand{\Pthen}{{\rm\bf then }}
\newcommand{\Pendif}{{\rm\bf endif }}
\newcommand{\Pelse}{{\rm\bf else }}
\newcommand{\Pendelse}{{\rm\bf endelse }}
\newcommand{\Preturn}{{\rm\bf return }}
\newcommand{\Real}{{\rm{I\hspace*{-0.4ex}R}}}
\newcommand{\Nat}{{\rm{I\hspace*{-0.4ex}N}}}
\newcommand{\Bool}{{\rm{I\hspace*{-0.4ex}B}}}
\newcommand{\Uint}{{\rm{I\hspace*{-0.5ex}I}}}
\newcommand{\Zet}{{\rm\sf Z\hspace*{-1.0ex}Z}}
\newcommand{\QuestEq}{\stackrel{?}{=}}
\newcommand{\DefEq}{:=}
\newcommand{\DisUnion}{\stackrel{.}{\cup}}
\newcommand{\IsPref}{\le_{\mbox{\tiny pr"af}}}


\title{
{\LARGE Piper-ME}\\
}

\maketitle

\chapter{Piper-ME}

Piper-ME is a port of Yacas to the Java Mobile Edition platform, CLDC version 1.1 
(that is, including double precision floating point numbers). The port is, however,
not finished yet. Most classes from the standard libraries which are nor part of
Java ME have been replaced by simplified reimplementation (in particular hashmaps
and extensdible arrays) or taken from the classpath project (big numbers).

\chapter{Integration to Mobile Applications}

\end{document}
