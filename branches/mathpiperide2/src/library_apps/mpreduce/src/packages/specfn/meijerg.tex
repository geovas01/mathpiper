\documentstyle[11pt,reduce]{article}
\title{{\tt meijerG}, a package for simplification\\
of Meijer's G function}
\date{}
\author{Victor S. Adamchik\\
	Wolfram Research Inc. \\
	former address : \\
	Byelorussian University, Minsk, Byelorussia\\
\\
\\
	Present \REDUCE{} form by \\
	Winfried Neun \\
	ZIB Berlin \\
        Email: {\tt Neun@sc.ZIB-Berlin.de}}
\begin{document}
\maketitle

This note describes the {\tt meijerG} package of \REDUCE{}, which is able
to do simplification of several cases of Meijer's G function. 
The simplifications are performed towards polynomials, elementary or
special functions or (generalized) hypergeometric functions.
Therefore this package should be used together with the \REDUCE{}
special function and hypergeometric (ghyper) package.

\section{Introduction}

The function 

\begin{displaymath}
G_{p q}^{m n} \left( z \  \Bigg\vert \  {(a_p) \atop (b_q)} \right)
\end{displaymath}

has been studied by C.~S.~Meijer beginning in 1936 and has been
called Meijer's G function later on. The complete definition of Meijer's
G function can be found in \cite{Prudnikov:90}.
Many well-known functions can be written as G functions,
e.g. exponentials, logarithms, trigonometric functions, Bessel functions
and hypergeometric functions.

Several hundreds of particular values can be found in \cite{Prudnikov:90}.

\section{\REDUCE{} operator {\tt meijerg}}

The operator {\tt meijerg} expects 3 arguments, namely the 
list of upper parameters (which may be empty), the list of lower
parameters (which may be empty too), and the argument.

The first element of the lists has to be the list of the
first m or n resp. parameter, e.g.

To describe 
\begin{displaymath}
G_{1 1}^{1 0} \left( x \  \Bigg\vert \  {1 \atop 0} \right)
\end{displaymath}

one has to write 
\begin{verbatim}

MeijerG({{},1},{{0}},x); % and the result is:

 HEAVISIDE( - X + 1)
---------------------
      GAMMA(1)

\end{verbatim}
and for
\begin{displaymath}
G_{0 2}^{1 0} \left( \frac{x^2}{4} \  \Bigg\vert \ {} \atop  {1+ \frac{1}{4} }
{1-\frac{1}{4}} \right)
\end{displaymath}
\begin{verbatim}

MeijerG({{}},{{1+1/4},1-1/4},(x^2)/4) * sqrt pi;


                   1      2
 SQRT(PI)*BESSELJ(---,X)*X
                   2
----------------------------
             4

\end{verbatim}

Note: Using the special function package these results will be
simplified further.

\begin{thebibliography}{9}

\bibitem{Prudnikov:90} A.~P.~Prudnikov, Yu.~A.~Brychkov, O.~I.~Marichev,
{\em Integrals and Series, Volume 3: More special functions},
Gordon and Breach Science Publishers (1990).

\end{thebibliography}
\end{document}
